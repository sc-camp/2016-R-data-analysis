\begin{frame}{R?}

R (pronounced aRrgh -- pirate style) is a programming language and
environment for statistical computing and graphics

\begin{itemize}
\item
  oriented towards data handling analysis and storage facility
\item
  R Base
\item
  Packages tools and functions (user contributed)
\item
  R Base and most R packages are available from the
  \href{cran.r-project.org}{Comprehensive R Archive Network (CRAN)}
\item
  Use R console or IDE: \textbf{Rstudio}, Deducer, vim/emacs\ldots{}
\item
  Comment is \textbf{\#}, help is \textbf{?} before a function name
\end{itemize}

\end{frame}

\begin{frame}[fragile]{Using R}

\begin{block}{\textbf{Installing/using packages}}

Install and load the \texttt{ggplot2} package (even if already
installed)

\begin{verbatim}
install.packages("ggplot2")
library(ggplot2)
\end{verbatim}

Or in one step, install if not available then load:

\begin{verbatim}
require(ggplot2) || {install.packages("ggplot2");
                     require(ggplot2)}
\end{verbatim}

\end{block}

\end{frame}

\begin{frame}[fragile]{Numbers in R: NaN and NA}

\begin{itemize}
\item
  NAN (not a number)
\item
  NA (missing value)
\end{itemize}

Basic handling of missing values

\begin{verbatim}
> x
[1]  1  2  3  4  5  6  7  8 NA
> mean(x)
[1] NA
> mean(x,na.rm=TRUE)
[1] 4.5 
\end{verbatim}

\end{frame}

\begin{frame}{Objects in R}

Objects in R obtain values by assignment.

This is achieved by the gets arrow, \textless{}-, or the equal sign, =.

Objects can be of different kinds.

\end{frame}

\begin{frame}{Data Structures}

\begin{itemize}
\item
  scalar:

  s = 3.14
\item
  vector:

  v = c(1, 2, ``ron'')
\item
  list:

  l = list(1:10, `a', pi)
\item
  matrix:

  m = matrix(seq(1:6), 2)
\item
  \textbf{dataframe}:

  df = data.frame(``col1'' = seq(1:4), ``col2'' = c(5, 6, ``cerveza'',
  6*7))
\item
  \ldots{}
\end{itemize}

\end{frame}

\begin{frame}[fragile]{Functions}

Functions are created using the function() directive and are stored as R
objects just like anything else. In particular, they are R objects of
class ``function''.

\begin{verbatim}
f <- function(<arguments>) {
## Do something interesting
}
\end{verbatim}

Functions in R are ``first class objects'', which means that they can be
treated much like any other R object.

The return value of a function is the last expression in the function
body to be evaluated.

\end{frame}

\begin{frame}[fragile]{Arguments}

R functions arguments can be matched positionally or by name. So the
following calls to sd are all equivalent

\begin{verbatim}
> mydata <- rnorm(100)
> sd(mydata)
> sd(x = mydata)
> sd(x = mydata, na.rm = FALSE)
> sd(na.rm = FALSE, x = mydata)
> sd(na.rm = FALSE, mydata)
\end{verbatim}

Even though it's legal, I don't recommend messing around with the order
of the arguments too much, since it can lead to some confusion.

\end{frame}

\begin{frame}[fragile]{Argument Matching}

You can mix positional matching with matching by name. When an argument
is matched by name, it is ``taken out'' of the argument list and the
remaining unnamed arguments are matched in the order that they are
listed in the function definition.

e.g.~The following two calls are equivalent.

\begin{verbatim}
lm(data = mydata, y ~ x, model = FALSE, 1:100)
lm(y ~ x, mydata, 1:100, model = FALSE)
\end{verbatim}

\end{frame}

\begin{frame}{Argument Matching}

\begin{itemize}
\item
  Most of the time, named arguments are useful on the command line when
  you have a long argument list and you want to use the defaults for
  everything except for an argument near the end of the list
\item
  Named arguments also help if you can remember the name of the argument
  and not its position on the argument list (plotting is a good
  example).
\end{itemize}

\end{frame}

\begin{frame}[fragile]{Defining a Function}

\begin{verbatim}
f <- function(a, b = 1, c = 2, d = NULL) {
}
\end{verbatim}

In addition to not specifying a default value, you can also set an
argument value to NULL.

\end{frame}

\begin{frame}[fragile]{Lazy Evaluation}

Arguments to functions are evaluated lazily, so they are evaluated only
as needed.

\begin{verbatim}
f <- function(a, b) {
a^2
}
f(2)
\end{verbatim}

This function never actually uses the argument b, so calling f(2) will
not produce an error because the 2 gets positionally matched to a.

\end{frame}

\begin{frame}[fragile]{Lazy Evaluation}

Another example

\begin{verbatim}
f <- function(a, b) {
print(a)
print(b)
}
> f(45)
[1] 45
Error in print(b) : argument "b" is missing, 
with no default
>
\end{verbatim}

Notice that ``45'' got printed first before the error was triggered.
This is because b did not have to be evaluated until after print(a).
Once the function tried to evaluate print(b) it had to throw an error.

\end{frame}

\begin{frame}[fragile]{Using R}

\begin{block}{\textbf{Usefull Functions}}

\begin{itemize}
\item
  List all objects in memory: \texttt{ls()}
\item
  Save an object: \texttt{save(obj,\ file)}
\item
  Load an object: \texttt{load(file)}
\item
  Set working directory: \texttt{setwd(dir)}
\end{itemize}

\end{block}

\end{frame}

\begin{frame}[fragile]{Entering Data}

\begin{block}{Reading CSV or text files}

\begin{verbatim}
# comma separated values
dat.csv <- read.csv(<file or url>)
# tab separated values
dat.tab <- read.table(<file or url>, 
    header=TRUE, sep = "\t")
\end{verbatim}

\end{block}

\end{frame}

\begin{frame}[fragile]{Entering Data}

\begin{block}{Reading data from other software: Excel, SPSS\ldots{}}

Excel Spreadsheets -- need \texttt{xlsx} package

\begin{verbatim}
read.xlsx()
\end{verbatim}

SPSS and Stata both need the \texttt{foreign} package

\begin{verbatim}
dat.spss <- read.spss(<file or url>, 
                      to.data.frame=TRUE)
             
dat.dta <- read.dta(<file or url>)
\end{verbatim}

\end{block}

\end{frame}

\begin{frame}{Data Frames}

Most easy structure to use, have a matrix structure.

\begin{itemize}
\item
  \textbf{Observations} are arranged as \textbf{rows} and
  \textbf{variables}, either numerical or categorical, are arranged as
  \textbf{columns}.
\item
  Individual rows, columns, and cells in a data frame can be accessed
  through many methods of indexing.
\item
  We most commonly use \textbf{object{[}row,column{]}} notation.
\end{itemize}

\end{frame}

\begin{frame}[fragile]{Accessing Items in a \texttt{data.frame}}

Aside with R are provided example datasets, i.e. \texttt{mtcars} that
can be used

\begin{verbatim}
data(mtcars)
head(mtcars)
colnames(mtcars)

# single cell value
mtcars[2,3]
# omitting row value implies all rows
mtcars[,3]
# omitting column values implies all columns
mtcars[2,]
\end{verbatim}

\end{frame}

\begin{frame}[fragile]{Accessing Items in a \texttt{data.frame}}

We can also access variables directly by using their names, either with
\textbf{object{[},``variable''{]}} notation or \textbf{object\$variable}
notation.

\begin{verbatim}
# get first 10 rows of variable `mpg` using two methods:
mtcars[1:10, "mpg"]
mtcars$mpg[1:10]
\end{verbatim}

\end{frame}

\begin{frame}{Exploring Data}

\begin{block}{Description Of Dataset}

\begin{itemize}
\item
  Using \textbf{dim}, we get the number of observations(rows) and
  variables(columns) in the dataset.
\item
  Using \textbf{str}, we get the structure of the dataset, including the
  class(type) of all variables.

  dim(mtcars) str(mtcars)
\item
  \textbf{summary} when used on a dataset, returns distributional
  summaries of variables in the dataset.

  summary(mtcars)
\item
  \textbf{quantile} function enables to get statistical metrics on the
  selected data

  quantile(mtcars\$mpg)
\end{itemize}

\end{block}

\end{frame}

\begin{frame}{Exploring Data}

\begin{block}{Conditional Exploration}

\begin{itemize}
\item
  \textbf{subset} enables to explore data conditionally

  subset(mtcars, cyl \textless{}= 5)
\item
  \textbf{by} enables to call a particular function to sub-groups of
  data

  by(mtcars, mtcars\$cyl, summary)
\end{itemize}

\end{block}

\end{frame}
